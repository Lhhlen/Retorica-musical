\documentclass[11pt,a4paper]{article}
\usepackage[utf8]{inputenc}
\usepackage{amsmath}
\usepackage{amsfonts}
\usepackage{amssymb}
\usepackage{graphicx}
\usepackage[left=2cm,right=2cm,top=2cm,bottom=2cm]{geometry}
\title{Figuras Retóricas}
\author{Análisis II}
\date{Enero, 2023}
\begin{document}
	\maketitle
	\tableofcontents
	\section{Figuras que afectan a la melodía}
	\subsection{De repetición / imitación}
		 \subsubsection{Dónde se produce la repetición}
			\begin{itemize}
				\item \textbf{Anáfora:} Repetición de un elemento musical al principio de distintas unidades.
				\item \textbf{Epístrofe:} Repetición de un elemento musical al final de distintas unidades.
				\item \textbf{Anadiplosis:} Motivo que cierra una unidad musical y se empieza con él la siguiente.
				\item \textbf{Epizeusis:} Repetición inmediata de un mismo motivo musical.
				\item \textbf{Epanadiplosis/Complexio:} Se repite un mismo elemento al principio y al final de la unidad musical, da la sensación de círculo.
			\end{itemize}
		\subsubsection{Modo en el que se produce la repetición o cómo opera}
			\begin{itemize}
				\item \textbf{Palillogia:} Repetición exacta de un elemento (=voz)
				\item \textbf{Sinonimia:} Repetición transportada de un elemento (transp, =voz)
				\item \textbf{Gradatio:} Igual que la sinonimia pero hace más repeticiones hasta formar una secuencia (transp secuencia, =voz). Si la gradatio es un movimiento ascendente se llama clímax.
				\item \textbf{Paronomasia:} Repetición de un mismo elemento pero añadiendo más notas, repetición variada (=voz). Es lo mismo que las disminuciones barrocas.
				\item \textbf{Polyptotón:} Repite a la misma altura (=altura)
				\item \textbf{Nímesis:} Repite a distinta altura o transportado (Fuga)
			
			\end{itemize}
		Una figura puede ser clasificada tanto por su posición como por cómo opera.
		\subsection{Figuras de hipotiposis/descriptivas}
			\begin{itemize}
				\item \textbf{Pathopoeia:} Acumulación de cromatismos, pasaje disonante con función expresiva.
				\item \textbf{Anábasis/Catábasis:} Ascenso, descenso respectivamente con relación al texto o intención.
				\item \textbf{Circulatio:} Algún tipo de significado linguístico que se pinta como una oscilación para indicar circulación, fluidez...
				\item \textbf{Interrogatio:} Movimiento melódico que deja en suspenso, imitando interrogación, lo más típico es un ascenso de segunda al final de la frase, una semicadencia o una cadencia frigia.
				\item \textbf{Exclamatio:} Ascenso más grande de alguna nota, al final de la frase o en medio, con o sin texto, puede ser o no vocal. El modelo por antonomasia de la exclamatio es el salto de 6ºm ascendente pero puede ser cualquier otro salto normalmente mayor que una tercera tanto consonante como disonante. Si es disonante es del subtipo \textbf{saltus duriusculus} (ejemplo: tritono, 7ºdism).
				
			\end{itemize}
	\section{Figuras que afectan a la armonía}
		\subsection{De disonancia:}
			Cómo desvía el discurso normal una disonancia, utilización expresiva de las disonancias, su tratamiento se sale de las reglas clásicas.
			\begin{itemize}
				\item \textbf{Anticipatio:} Traerse una de la armonía siguiente que forma disonancia.
				\item \textbf{Transitus notae:} Nota de paso. Cuando la nota(s) es cromática: \textbf{Pasus duriusculae}.
				\item \textbf{Superyecto:} Escapada superior (2º y luego 3º)
				\item \textbf{Subsumptio:} Escapada inferior.
				\item \textbf{Syncope/sincopatio:} Retardo normal.
				\item \textbf{Ellipsis:} Nota cercana (3º y luego 2º)
				\item \textbf{Heterolepsis:} Disonancia tomada sin preparación, más heavy que la ellipsis, salto hacia una nota disonante y luego normalmente va por grados conjuntos.
				\item \textbf{Catachresis:} La NO resolucicón de una disonancia.
				\item \textbf{Parrhesia:} Falsa relación armónica.Puede ser cromática o tritono. Enlace de armonías que no se enlazan como V y IV: queda tritono, puede generar una relación extraña en el oído, no es una disonancia que se de en un punto sino que se da en dos armonías consecutivas: Siempre es en dos voces distintas porque las voces hacen conducción normal.
				\item \textbf{Cadencia duriusculae:} Disonancias fuertes que ocurren junto con los acordes de una cadencia.
			\end{itemize}
		\subsection{De acordes:}
			\begin{itemize}
				\item \textbf{Noema:} Pasaje vertical dentro de un contexto contrapuntístico: Interrupción del tejido contrapuntístico.
				\subitem \textbf{Analepsis:} Repetición exacta del noema a cotinuación.
				\subitem \textbf{Mímesis noema:} Repetición transportada o con alguna modificación del noema a continuación.
				\subitem \textbf{Anadiplosis noema:} Si el mímesis noema se repite a continuación.\\
				No confundir con pathopoeia si el pasaje es disonante, los noemas suelen ser consonantes.
				
			\end{itemize}
		\section{Figuras que afectan a varios parámetros:}
			\subsection{Por adición:} 
				\begin{itemize}
					\item \textbf{Variatio (Passagio):} Desarrollo virtuosístico u ornamental (disminuciones-paronomasia), se hace para destacar una palabra. Podía comprender distintos tipos de ornamentos:
					\subitem \textbf{Trillo:} Trino.
					\subitem \textbf{Tremolo:} Trino lento, repetición de dos notas conjuntas.
					\subitem \textbf{Triata:} Varias notas conjuntas seguidas muy rápidas son sentido ornamental.\\ Pueden no ser solo figuras ornamentales, pueden ser también figuras de hipotiposis.
					\item \textbf{Paréntesis:} Interrupción momentánea del discurso para insertar un pasaje contrastante.
					\item \textbf{Suspensio:} Interrupción o detención del discurso musical como un calderón de manera suspendida, no confundir con el abruptio.
					\item \textbf{Abruptio (Aposiopesis):} Final súbito o interrupción imprevista, de manera abrupta, normalmente es una figura de silencio pero también puede ser que aparezca una cosa nueva.
					\item \textbf{Prosopopeia:} Personificación o imitación de otra cosa que no sea un ser humano, también se incluyen onomatopeyas.					
				\end{itemize}
			\subsection{Pors sustracción:}
				\begin{itemize}
					\item \textbf{Suspiratio:} Inserción de silencio a contratiempo con valor expresivo: Ha salido de la música vocal.
				\end{itemize}
			\subsection{Por permutación:}
				\begin{itemize}
					\item \textbf{Mutatio Toni:} Cambio de modo (mayor - menor) con finalidad expresiva.
					\item \textbf{Hypallage:} Imitación por movimiento contrario.
					\item \textbf{Antimetabole:} Imitación por retrogradación.
					\item \textbf{Hyperbaton:} Cambiar motivos de sitio creando efecto sorpresa.
					\item \textbf{Metábasis:} Cruce de voces o partes (las voces las llaman partes).
					\item \textbf{Hipérbole:} Todo aquello que se salga del discurso normal, un salto muy alto: Exageración, sorpresa.
					\item \textbf{Prólogo:} Nota de paso que se estira más de lo esperado.
				\end{itemize}
\end{document}